%% Current Release: v.0.1
%% Working Version: v.0.1.1


\documentclass[12pt]{article}


%% Formatting
\usepackage[margin = 1in]{geometry}
\usepackage[indent = 0in, skip = 0.25\baselineskip]{parskip}
	% \pagestyle{empty}
% \usepackage{indentfirst}

\usepackage{fontspec}
    \setmainfont{Charis SIL}
	\setmonofont{Fira Mono}

%% Section Headers
\usepackage[explicit]{titlesec}
	\titleformat{\section}{\bf\normalsize}{\thesection \quad #1}{0pt}{\vspace*{-.5\baselineskip}}
	\titleformat{\subsection}{\it\normalsize}{\textbf{\thesubsection} \quad \textbf{#1}}{0pt}{\vspace*{-.5\baselineskip}}
	\titleformat{\subsubsection}{\it\normalsize}{\thesubsubsection \quad #1}{0pt}{\vspace*{-.5\baselineskip}}

	\titleformat{name =\section, numberless}{\bf\normalsize}{#1}{0pt}{\vspace*{-.5\baselineskip}}
	\titleformat{name =\subsection, numberless}{\it\normalsize}{\textbf{#1}}{0pt}{\vspace*{-.5\baselineskip}}
	\titleformat{name =\subsubsection, numberless}{\it\normalsize}{#1}{0pt}{\vspace*{-.5\baselineskip}}

	% Enable for "Part" Section Level above Section:
	
		% \titleclass{\part}{straight}[\chapter]
		% \renewcommand{\thepart}{\Roman{part}}

		% \titleformat{\part}{\bf\normalsize}{\underline{Part \thepart \quad #1}}{0pt}{\setcounter{section}{0}}

		% \titleformat{name =\part, numberless}{\bf\normalsize}{\underline{#1}}{0pt}{}

%% Outline

\usepackage{outlines}
    % \renewcommand{\outlinei}{enumerate}
    % \renewcommand{\outlineii}{enumerate}
    % \renewcommand{\outlineiii}{enumerate}
    % \renewcommand{\outlineiiii}{enumerate}

%% Title 
\makeatletter
    \newcommand{\course}[1]{\def\@course{#1}}
	
	\renewcommand{\maketitle}{%
	\noindent\@author\\
	\@date
	\begin{center}
		\textbf{\@title}
	\end{center}
	\vspace*{\baselineskip}
} 
    \newcommand{\maketitlecourse}{%
		\renewcommand{\maketitle}{%
		\noindent\@author\\
		\@course\\
		\@date
		\begin{center}
			\textbf{\@title}\\
		\end{center}
		\vspace*{\baselineskip}
		} 
} 
\makeatother

%% Comments/Notes
\usepackage[ % to do notes in document
	size=scriptsize, 
	tickmarkheight=0.2cm,
	textwidth = 2in,
	color=lightgray, 
	textcolor=black,
	colorinlistoftodos,
	obeyFinal]{todonotes}

	\usepackage[dvipsnames]{xcolor}
	\todostyle{purp}{backgroundcolor=Thistle,linecolor=Thistle}
	\todostyle{red}{color=WildStrawberry}
	\todostyle{blue}{color=Cerulean}
	\todostyle{green}{color=PineGreen}

%%%% Packages %%%%%%%%%%%%%%%%%%%%%%%%%%%%%%%%%%%%%%
\usepackage[hidelinks]{hyperref}


%%%% Titling %%%%%%%%%%%%%%%%%%%%%%%%%%%%%%%%%%%%%%%
\author{Vincent N. Mariani\\vmariani@udel.edu}
\date{\today}
\course{}
\title{LaTeX Basic Template (v0.1)}
%%%% Begin Document %%%%%%%%%%%%%%%%%%%%%%%%%%%%%%%%
\begin{document}
    \maketitle%course
%%%% Body %%%%%%%%%%%%%%%%%%%%%%%%%%%%%%%%%%%%%%%%%%

This folder contains the essential files for a basic paper in \LaTeX. 

\begin{outline}
    \0 Files:
        \1 README.pdf
        \1 preamble.tex
        \1 root.tex
\end{outline}


\tableofcontents

\section{Intro to \LaTeX}

\subsection{How does this work?}

\subsection{General Useful Commands}

\begin{outline}[itemize]
    \1 Insert Spacing
        \2 \texttt{\textbackslash vspace\{l\}} and \texttt{\textbackslash hspace\{l\}}
            \3 Inserts a vertical or horizontal space of length \{l\}.
        \2 New line : \texttt{\textbackslash\textbackslash}
            \3 Two backslashes make a line break

    \1 Text Styles
        \2 \texttt{\textbackslash textbf}: \textbf{bold}
        \2 \texttt{\textbackslash textit}: \textit{italic}
    \1 Lengths
        \2 \texttt{ex}: x-height. 
        \2 \texttt{em}: m-width
        \2 \texttt{in, cm, pt}: number in regular units
        \2 \texttt{\textbackslash baselineskip}:  Line height
            \3 can be preceeded by a number for a multiple / a negative
    \1 Other Commands
\end{outline}

\section{The Preamble}

\subsection{What's a preamble?}

The preamble to a LaTeX document consists of everything before \texttt{\textbackslash begin{document}} is called, and must begin with a \texttt{\textbackslash documentclass\{\}} call. The default document class in this template is \texttt{article}, but you may change it as you wish.

\subsection{Using preamble.tex}

In this folder, there is a ``preamble.tex'' file that contains formatting and layout packages and settings. This bit of the README goes over what's in there: 


\subsection{Packages and Commands Explained}

Packages: \texttt{geometry, parskip, indentfirst, fontspec}

\subsubsection{geometry} \texttt{geometry}

    The package \texttt{geometry} provides settings related to page layout and size. The default setting for this template is \texttt{[margin = 1in]}, providing 1in margins all around. 

    Some useful package options to know are: 

    \begin{itemize}
        \item \texttt{left, right, top, bottom}: These will set the margins individually. Use the same syntax as \texttt{[margin = 1in]}, so \texttt{[left = 1in]}
        \item \texttt{paperwidth, paperheight}: Sets custom page size
    \end{itemize}

\subsubsection{parskip}

    The package \texttt{parskip} sets the paragraph indent size and skip size between paragraphs. 

\subsubsection{setspace}
    The package \texttt{setspace} enables easy line-spacing commands that can be called anywhere in the document: \texttt{\textbackslash singlespacing, \textbackslash onehalfspacing,} and \texttt{\textbackslash doublespacing}. 

\subsubsection{fontspec}

    The package \texttt{fontspec} is a powerful font engine for \LaTeX. Do note that you need to compile using either XeLaTeX or LuaLaTeX for this to work; pdfLaTeX will crash. 

    The template sets ``Charis SIL'' as the default font, which is a nice serif font with support for the full IPA. Note that 11pt Charis is similar to 12pt Times New Roman. If you're running this on your computer (i.e., not Overleaf), you can sub this for any installed font on your system. 

    Some useful package options:

    Some useful package commands:

\subsubsection{fancyhdr}

    The package \texttt{fancyhdr} allows for custom headers and footers. 

    This template defines two styles: ``lTitleRAuthorPage'' and ``rAuthorPage''. 
    
    ``lTitleRAuthorPage'' has the title of the document (set by the command \texttt{\textbackslash title\{\}} -- if the title is too long, set a different one just for the header with \texttt{\textbackslash shorttitle\{\}}), and the author's name (defaults to the name set by \texttt{\textbackslash author\{\}}, but just the last name can be shown by also setting \texttt{\textbackslash authorlast\{\}}) and page number on the right. ``rAuthorPage'' just has the author name and page number on the right. 

    The default page style for the template is provided by \LaTeX: the ``empty'' style - this has no headers or footers. For just a page number in the footer, you can use the ``plain'' style. You can also choose a different page style for a particular page (good fot title page) by using \texttt{\textbackslash thispagestyle\{\}}

    If you need to define your own style, you can just redo the code in the preamble to make a new one, and use \texttt{\textbackslash pagestyle\{\}} to set that style. 

\subsubsection{titlesec}

    The package \texttt{titlesec} sets the formatting for section headers, called with the commands \texttt{\textbackslash section\{\}, \textbackslash subsection\{\}, \textbackslash subsubsection\{\}}. Each section is numbered. For an unnumbered section, use the starred variant of the command, as in \texttt{\textbackslash section*\{\}}.

    There is a commented out definition for a ``part'' heading level, if you wish to use that. 


\subsubsection{outlines}

    The \texttt{outlines} package defines an \texttt{outline} environment which can be used to create nested lists easily. To make an outline, use the following code:

    \texttt{\textbackslash begin{outline}
                \textbackslash 0 This makes a normal paragraph
                    \textbackslash 1 Level 1
                        \textbackslash 2 Level 2
                            \textbackslash 3 Level 3
                                \textbackslash 4 Level 4
            \textbackslash end{outline}
    }

    Which produces: 

        \begin{outline}
            \0 This makes a normal paragraph
                \1 Level 1
                    \2 Level 2
                        \3 Level 3
                            \4 Level 4
        \end{outline}



    The template default is to use the \texttt{enumerate} environment, but you can change the list type to \texttt{itemize}. Alternatively, you can use \texttt{\textbackslash begin{outline}[itemize]} to make a single list bulleted. You can also do the opposite by making\texttt{itemize} default and using \texttt{[enumerate]} for the individual list.

\subsubsection{Titling}

    The section labeled ``Titling'' is used to define the style of the \texttt{\textbackslash maketitle} command, and to define commands to set the author's last name, course name (line 2 of the header) and short document title. 

    One thing to notice is the commands \texttt{\textbackslash makeatletter} and \texttt{\textbackslash makeatother}. Each character in \LaTeX is assigned a character class, which the compiler uses to determine what's a command and what's not. Changing \@ to a letter lets us define some additional commands that won't be able to be called when \@ is set to the other class. For the vast majority of things outside of designing packages and setting certain commands, you won't need this. 

\subsubsection{todonotes}

    The package \texttt{todonotes} allows you to put notes into the margin of a document. I've set some defaults for the style of the notes, as well as defining four color options that can be called with \texttt{\textbackslash todo[purp]{This is the note.}}\todo[purp]{This is the note.} 

    Notes by default will appear in the right margin, 15pts from the edge of the page. You can change that by adjusting the definition of \texttt{\textbackslash todowidth} command. You can also use the ``noline'' option to not draw a line between the margin and the text, or the ``inline'' option to place the todonote on the page like this:

    \texttt{\textbackslash todo[red, inline]{This is an inline note.}}
    \todo[red, inline]{This is an inline note.}

\section{Useful Packages}

\subsubsection{indentfirst}

        The \texttt{indentfirst} package sets the first paragraph of the paper/section to be indented on the first line like the other paragraphs. 

%%%% End Document %%%%%%%%%%%%%%%%%%%%%%%%%%%%%%%%%%
\end{document}